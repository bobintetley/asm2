%% LyX 1.6.2 created this file.  For more info, see http://www.lyx.org/.
%% Do not edit unless you really know what you are doing.
\documentclass[english]{article}
\usepackage[T1]{fontenc}
\usepackage[latin9]{inputenc}

%%%%%%%%%%%%%%%%%%%%%%%%%%%%%% Textclass specific LaTeX commands.
\newenvironment{lyxcode}
{\par\begin{list}{}{
\setlength{\rightmargin}{\leftmargin}
\setlength{\listparindent}{0pt}% needed for AMS classes
\raggedright
\setlength{\itemsep}{0pt}
\setlength{\parsep}{0pt}
\normalfont\ttfamily}%
 \item[]}
{\end{list}}

%%%%%%%%%%%%%%%%%%%%%%%%%%%%%% User specified LaTeX commands.
\usepackage[T1]{fontenc}
\usepackage{ae,aecompl}

\usepackage{babel}

\begin{document}

\title{FAQ}

\maketitle
\tableofcontents{}


\section{Questions and Answers}


\subsection{What's the default username and password?}

A. It's username {}``user'' with the password {}``letmein''.


\subsection{Can I have a demo of the software?}

A. No. This software is free - free as in gratis, free as in libre/freedom,
you can download it for free, you can give it your friends, you can
install it on as many computers as you like.


\subsection{My shelter/site is very interested in your software! Can I have it
please?}

A. Yes, download it. Just download, install and go. If you like it,
buy me a beer.


\subsection{Can I phone you up and ask you some questions?}

A. No. I donate my time and effort for free in making this software
- I don't do phone support. If you want to chat about ASM or ask specific
questions, please direct them to the forums or mailing list (or email
me directly - db@sheltermanager.com).


\subsection*{.. please, I can't understand your website.}

No offence, but if you're too lazy to help yourself, I'm too lazy
to help you. Go find someone who knows a bit about computing and get
them to help you (and read this site).


\subsection{This is really free? Are you mad?}

A. Yes and no, I'm not mad. I don't get anything for doing ASM and
I devote my spare time to doing it. All I ask is that if you like
and use the software you might consider donating something or helping
out. I'll never make a living doing this, since by definition my userbase
has no money, but even if you just give me enough to a buy a beer
or a pizza by way of saying thank you, it would be much appreciated.


\subsection{Help! How do I {[}something{]} (eg: change animal codes, create a
diary note, etc.)}

Consult the ASM user manual ( http://sheltermanager.sf.net/help )


\subsection{How can I brand reports for my shelter?}

A. Reports in Animal Shelter Manager are generated as HTML for viewing
by web browsers. The internal report viewer in Animal Shelter Manager
is actually an HTML browser. When constructing reports, it uses a
header file and a footer file to surround the report content. These
two files, header.dat and footer.dat are located in the reports directory
of the media files. 

They are both plain text HTML files which can be edited to change
the style and content of reports. You may even use images and brand
your reports with your company logo - ASM will automatically retrieve
the image from the media directory and integrate it into resulting
reports for you.

You can access these files by going to System->Media Files, picking
the reports directory, highlighting these files in turn and hitting
the {}``download'' button to save them to your hard disk. There,
you can edit them and then use the {}``upload'' button to reupload
them back to the media server.

Use this same technique to edit the files under the templates directory
accordingly for your shelter.


\subsection{How do I back my data up?}

A. If you are using ASM's local (HSQLDB) database, just copy all the
files starting with {}``localdb'' in Documents and Settings/{[}your
user name{]}/.asm/ to another disk/CD/whatever. If you are using MySQL
or PostgreSQL, see the manuals for those products. To restore, or
put the data on another machine, copy your backup back to the same
place you restored from.


\subsection{How do I network my data?}

A. If you're using the default, local database, you can do the following
to allow other ASM clients on your network to use your data:
\begin{enumerate}
\item Make sure ASM is closed on your computer
\item Run the {}``Local Database Server'' shortcut from Programs->Animal
Shelter Manager (for Windows, MacOSX and Unix users can run the run\_hsqlserver.sh
script). Note, you will have to run this each time you start up your
computer - you can copy this shortcut to Programs->Startup to have
Windows do it for you. Once you've started the database server, you
can open ASM on the computer running the server if you want.
\item Install ASM on your other computers. When asked where the database
is, hit the Scan button at the bottom right - after a short wait,
details of your server will be completed and you can hit Ok to connect
and continue (ASM will remember and use this database in future).
If you already had ASM installed with a local database, use the {}``Switch
Database'' option under the Preferences menu to switch to your database
server and use the Scan button.
\end{enumerate}

\subsection{How can I brand my published internet sites?}

A. Internet publishing is handled in almost the same way as reports,
except the files are located in the internet directory and called
pih.dat (header) and pif.dat (footer). There is also an additional
file located here called pib.dat. This represents the body of each
animal entry in the published web pages and as well as changing the
layout, you can also change the information shown by using the standard
substitution strings used by the document publisher.


\subsection{How can I create my own document templates?}

A. Document templates are stored in your database. You can browse
the tree of media files by going to the System menu, then picking
{}``Media Files''. 

You can then download an existing template (from the template directory),
change it and reupload or simply upload new templates.


\subsection{Do I have to use the inbuilt report viewer?}

A. As reports are generated as HTML, you are free to use an external
browser for viewing and printing them instead of Animal Shelter Manager's
built in report viewer. To do this, go to the Preferences->Settings
menu option and untick the \textquotedbl{}Use internal browser\textquotedbl{}
box. ASM will open reports with whatever association it has for html
files (see System->Configure File Types)


\subsection{How can I change a file association?}

A. Filetypes are configurable from the System->Configure File Types
option.


\subsection{We want the numbering scheme to fit in with our existing data and
use higher numbers - can ASM do this?}

A. Yes - every code ASM generates it looks at the existing data for
the previous highest so all you have to do is enter an animal and
manually overwrite it's code with the number you'd like to start at
- ASM will carry on counting thereafter.


\subsection{Can I store other information in ASM?}

A. The media tab allows you store any type of document with any type
of record. This could be used to attach weight charts, medical information,
etc. to animals. ASM offers an edit button to allow you to edit these
documents with their appropriate package and will automatically store
them on the server again when you are finished.

Note that there is a limit on size - you cannot store files larger
than 8Mb as a media attachment.

You can also use the Additional Fields option under the System menu
to add your own fields to the animal and owner screens.

The log tab allows ongoing addition of other types of information
with dates (this is good for recording things like complaints, bite
reports, email trails, etc).


\subsection{What development environment do you use?}

A. I develop under Debian Linux, using Sun JDK 1.6.x, Ant (http://ant.apache.org)
and GNU Make as the build tool and Vim (http://www.vim.org) for editing
the code. If you want to help, or hack about with the code yourself,
feel free to use anything you like. An Eclipse project is included
if you'd prefer to use that.


\subsection{How do you manage the project?}

A. With great difficulty - I'm just one guy and I get some great help
from volunteers, but since ASM doesn't pay the bills, I fit it in
around a full time job.


\subsection{We're not happy with the wording and terminology of some of the items
in ASM - can we change them?}

A. Yes - see the translation HOWTO under the Documentation link above.
No, I won't incorporate them into the main program as most ASM users
are familiar with the existing terminology.




\subsection{Can I pay you to make changes to ASM for our shelter?}

A. Certainly! Contact me for a quote - db@sheltermanager.com


\subsection{Can I use MySQL for my database?}

A. Yes. The procedure for Windows users is:
\begin{enumerate}
\item Download and install MySQL from www.mysql.com. It will set up Windows
services for the databases, etc. etc.\\

\item Copy the mysql.sql file from C:\textbackslash{}Program Files\textbackslash{}ASM\textbackslash{}data\textbackslash{}sql
to C:\textbackslash{}\\

\item Open a MySQL console. If there's no start menu item for it, go to
C:\textbackslash{}Program Files\textbackslash{}MySQL\textbackslash{}bin
and double click \textquotedbl{}mysql\textquotedbl{}\\

\item Enter the following SQL commands at the mysql> prompt: \\
\\
\emph{CREATE DATABASE asm; }\\
\emph{USE asm; }\\
\emph{SOURCE c:\textbackslash{}mysql.sql }\\
\emph{(lots of output will be displayed) }\\
\emph{GRANT ALL PRIVILEGES ON {*}.{*} TO root@'\%'; }\\
\emph{QUIT;}\\

\item Edit the file: \\
\\
\emph{C:\textbackslash{}Documents and Settings/YOURUSER/.asm/jdbc.properties}\\

\item Replace any existing contents of the file with this line:\\
\\
\emph{JDBCURL=jdbc:mysql://localhost/asm?user=root\&zeroDateTimeBehaviour=convertToNull}\\

\item For each machine on your network, substitute \emph{\textquotedbl{}localhost\textquotedbl{}}
in the above line of that file for the IP address of the machine running
MySQL.
\end{enumerate}
For Linux/other platform users the procedure is the same, except you
should install MySQL from your distribution and {}``cd'' to the
asm/data/sql directory and just use {}``SOURCE mysql.sql'' when
executing the database commands. Documents and Settings can be replaced
with \$HOME.


\subsection{Can I use PostgreSQL for my database?}

A. Yes. In the ASM installation directory, under data/sql, you'll
find a postgresql.sql file containing the database schema for PostgreSQL.
I'm going to assume if you're wanting to use PostgreSQL that you know
what you're doing.


\subsection{Can I use SSL with PostgreSQL?}

A. Yes. You'll need to modify the jdbc.properties file. Add the following
parameters to the end of the URL:
\begin{lyxcode}
\&ssl=true\&sslfactory=org.postgresql.ssl.NonValidatingFactory
\end{lyxcode}
This enables the encryption portion of SSL, but does not validate
your server certificate. If your server certificate is signed by a
trusted authority, then you can remove the sslfactory parameter.

If you want to validate the server certificate, or use client certificates
as well, read this page for further information:

http://jdbc.postgresql.org/documentation/80/ssl-client.html


\section{Troubleshooting}


\subsection{Help! I picked the wrong database option at the beginning and now
I can't change it!}

A. If you open Windows Explorer and look at Documents and Settings/{[}your
user name{]}/, (or Users/{[}your user name{]}/ for Vista/Windows 7)
you'll see a directory called {}``.asm''. Delete this directory
and restart ASM - this effectively resets ASM to be a brand new installation.

If you are a unix user, remove \$HOME/.asm


\subsection{When I try to generate documentation, I can't see any templates?}

A. Go to System->Options and make sure that you have chosen the correct
word processor - it's this that determines the file name extension
ASM will filter on when displaying templates.


\subsection{When attempting to view the printable help, I get an error {}``no
file type for pdf''?}

A. You need Adobe Acrobat installed. If you installed it after ASM,
you will need to go to Preferences->Configure File Types and hit the
rescan button to discover it.


\subsection{When attempting to create a document, I get an error {}``no file
type for odt''?}

A. You need OpenOffice.org installed to use the supplied templates
(visit http://www.openoffice.org to download it). Once you've done
this, go to the Preferences->Configure File Types screen and hit the
rescan button on right of the toolbar.

If you have Microsoft Office 97, 2000 or 2003 and would prefer to
use that, select {}``Rich Text Compatible'' as your word processor
under System->Options and go to Preferences->Configure File Types
and verify that the path to winword.exe is set against the rtf entry
(ASM should do this automatically when it is first run if Microsoft
Office is available). You can also use Microsoft Office 2007 natively
and templates are supplied.
\end{document}
